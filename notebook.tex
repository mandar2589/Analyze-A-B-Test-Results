
% Default to the notebook output style

    


% Inherit from the specified cell style.




    
\documentclass[11pt]{article}

    
    
    \usepackage[T1]{fontenc}
    % Nicer default font (+ math font) than Computer Modern for most use cases
    \usepackage{mathpazo}

    % Basic figure setup, for now with no caption control since it's done
    % automatically by Pandoc (which extracts ![](path) syntax from Markdown).
    \usepackage{graphicx}
    % We will generate all images so they have a width \maxwidth. This means
    % that they will get their normal width if they fit onto the page, but
    % are scaled down if they would overflow the margins.
    \makeatletter
    \def\maxwidth{\ifdim\Gin@nat@width>\linewidth\linewidth
    \else\Gin@nat@width\fi}
    \makeatother
    \let\Oldincludegraphics\includegraphics
    % Set max figure width to be 80% of text width, for now hardcoded.
    \renewcommand{\includegraphics}[1]{\Oldincludegraphics[width=.8\maxwidth]{#1}}
    % Ensure that by default, figures have no caption (until we provide a
    % proper Figure object with a Caption API and a way to capture that
    % in the conversion process - todo).
    \usepackage{caption}
    \DeclareCaptionLabelFormat{nolabel}{}
    \captionsetup{labelformat=nolabel}

    \usepackage{adjustbox} % Used to constrain images to a maximum size 
    \usepackage{xcolor} % Allow colors to be defined
    \usepackage{enumerate} % Needed for markdown enumerations to work
    \usepackage{geometry} % Used to adjust the document margins
    \usepackage{amsmath} % Equations
    \usepackage{amssymb} % Equations
    \usepackage{textcomp} % defines textquotesingle
    % Hack from http://tex.stackexchange.com/a/47451/13684:
    \AtBeginDocument{%
        \def\PYZsq{\textquotesingle}% Upright quotes in Pygmentized code
    }
    \usepackage{upquote} % Upright quotes for verbatim code
    \usepackage{eurosym} % defines \euro
    \usepackage[mathletters]{ucs} % Extended unicode (utf-8) support
    \usepackage[utf8x]{inputenc} % Allow utf-8 characters in the tex document
    \usepackage{fancyvrb} % verbatim replacement that allows latex
    \usepackage{grffile} % extends the file name processing of package graphics 
                         % to support a larger range 
    % The hyperref package gives us a pdf with properly built
    % internal navigation ('pdf bookmarks' for the table of contents,
    % internal cross-reference links, web links for URLs, etc.)
    \usepackage{hyperref}
    \usepackage{longtable} % longtable support required by pandoc >1.10
    \usepackage{booktabs}  % table support for pandoc > 1.12.2
    \usepackage[inline]{enumitem} % IRkernel/repr support (it uses the enumerate* environment)
    \usepackage[normalem]{ulem} % ulem is needed to support strikethroughs (\sout)
                                % normalem makes italics be italics, not underlines
    

    
    
    % Colors for the hyperref package
    \definecolor{urlcolor}{rgb}{0,.145,.698}
    \definecolor{linkcolor}{rgb}{.71,0.21,0.01}
    \definecolor{citecolor}{rgb}{.12,.54,.11}

    % ANSI colors
    \definecolor{ansi-black}{HTML}{3E424D}
    \definecolor{ansi-black-intense}{HTML}{282C36}
    \definecolor{ansi-red}{HTML}{E75C58}
    \definecolor{ansi-red-intense}{HTML}{B22B31}
    \definecolor{ansi-green}{HTML}{00A250}
    \definecolor{ansi-green-intense}{HTML}{007427}
    \definecolor{ansi-yellow}{HTML}{DDB62B}
    \definecolor{ansi-yellow-intense}{HTML}{B27D12}
    \definecolor{ansi-blue}{HTML}{208FFB}
    \definecolor{ansi-blue-intense}{HTML}{0065CA}
    \definecolor{ansi-magenta}{HTML}{D160C4}
    \definecolor{ansi-magenta-intense}{HTML}{A03196}
    \definecolor{ansi-cyan}{HTML}{60C6C8}
    \definecolor{ansi-cyan-intense}{HTML}{258F8F}
    \definecolor{ansi-white}{HTML}{C5C1B4}
    \definecolor{ansi-white-intense}{HTML}{A1A6B2}

    % commands and environments needed by pandoc snippets
    % extracted from the output of `pandoc -s`
    \providecommand{\tightlist}{%
      \setlength{\itemsep}{0pt}\setlength{\parskip}{0pt}}
    \DefineVerbatimEnvironment{Highlighting}{Verbatim}{commandchars=\\\{\}}
    % Add ',fontsize=\small' for more characters per line
    \newenvironment{Shaded}{}{}
    \newcommand{\KeywordTok}[1]{\textcolor[rgb]{0.00,0.44,0.13}{\textbf{{#1}}}}
    \newcommand{\DataTypeTok}[1]{\textcolor[rgb]{0.56,0.13,0.00}{{#1}}}
    \newcommand{\DecValTok}[1]{\textcolor[rgb]{0.25,0.63,0.44}{{#1}}}
    \newcommand{\BaseNTok}[1]{\textcolor[rgb]{0.25,0.63,0.44}{{#1}}}
    \newcommand{\FloatTok}[1]{\textcolor[rgb]{0.25,0.63,0.44}{{#1}}}
    \newcommand{\CharTok}[1]{\textcolor[rgb]{0.25,0.44,0.63}{{#1}}}
    \newcommand{\StringTok}[1]{\textcolor[rgb]{0.25,0.44,0.63}{{#1}}}
    \newcommand{\CommentTok}[1]{\textcolor[rgb]{0.38,0.63,0.69}{\textit{{#1}}}}
    \newcommand{\OtherTok}[1]{\textcolor[rgb]{0.00,0.44,0.13}{{#1}}}
    \newcommand{\AlertTok}[1]{\textcolor[rgb]{1.00,0.00,0.00}{\textbf{{#1}}}}
    \newcommand{\FunctionTok}[1]{\textcolor[rgb]{0.02,0.16,0.49}{{#1}}}
    \newcommand{\RegionMarkerTok}[1]{{#1}}
    \newcommand{\ErrorTok}[1]{\textcolor[rgb]{1.00,0.00,0.00}{\textbf{{#1}}}}
    \newcommand{\NormalTok}[1]{{#1}}
    
    % Additional commands for more recent versions of Pandoc
    \newcommand{\ConstantTok}[1]{\textcolor[rgb]{0.53,0.00,0.00}{{#1}}}
    \newcommand{\SpecialCharTok}[1]{\textcolor[rgb]{0.25,0.44,0.63}{{#1}}}
    \newcommand{\VerbatimStringTok}[1]{\textcolor[rgb]{0.25,0.44,0.63}{{#1}}}
    \newcommand{\SpecialStringTok}[1]{\textcolor[rgb]{0.73,0.40,0.53}{{#1}}}
    \newcommand{\ImportTok}[1]{{#1}}
    \newcommand{\DocumentationTok}[1]{\textcolor[rgb]{0.73,0.13,0.13}{\textit{{#1}}}}
    \newcommand{\AnnotationTok}[1]{\textcolor[rgb]{0.38,0.63,0.69}{\textbf{\textit{{#1}}}}}
    \newcommand{\CommentVarTok}[1]{\textcolor[rgb]{0.38,0.63,0.69}{\textbf{\textit{{#1}}}}}
    \newcommand{\VariableTok}[1]{\textcolor[rgb]{0.10,0.09,0.49}{{#1}}}
    \newcommand{\ControlFlowTok}[1]{\textcolor[rgb]{0.00,0.44,0.13}{\textbf{{#1}}}}
    \newcommand{\OperatorTok}[1]{\textcolor[rgb]{0.40,0.40,0.40}{{#1}}}
    \newcommand{\BuiltInTok}[1]{{#1}}
    \newcommand{\ExtensionTok}[1]{{#1}}
    \newcommand{\PreprocessorTok}[1]{\textcolor[rgb]{0.74,0.48,0.00}{{#1}}}
    \newcommand{\AttributeTok}[1]{\textcolor[rgb]{0.49,0.56,0.16}{{#1}}}
    \newcommand{\InformationTok}[1]{\textcolor[rgb]{0.38,0.63,0.69}{\textbf{\textit{{#1}}}}}
    \newcommand{\WarningTok}[1]{\textcolor[rgb]{0.38,0.63,0.69}{\textbf{\textit{{#1}}}}}
    
    
    % Define a nice break command that doesn't care if a line doesn't already
    % exist.
    \def\br{\hspace*{\fill} \\* }
    % Math Jax compatability definitions
    \def\gt{>}
    \def\lt{<}
    % Document parameters
    \title{Analyze\_ab\_test\_results\_notebook}
    
    
    

    % Pygments definitions
    
\makeatletter
\def\PY@reset{\let\PY@it=\relax \let\PY@bf=\relax%
    \let\PY@ul=\relax \let\PY@tc=\relax%
    \let\PY@bc=\relax \let\PY@ff=\relax}
\def\PY@tok#1{\csname PY@tok@#1\endcsname}
\def\PY@toks#1+{\ifx\relax#1\empty\else%
    \PY@tok{#1}\expandafter\PY@toks\fi}
\def\PY@do#1{\PY@bc{\PY@tc{\PY@ul{%
    \PY@it{\PY@bf{\PY@ff{#1}}}}}}}
\def\PY#1#2{\PY@reset\PY@toks#1+\relax+\PY@do{#2}}

\expandafter\def\csname PY@tok@w\endcsname{\def\PY@tc##1{\textcolor[rgb]{0.73,0.73,0.73}{##1}}}
\expandafter\def\csname PY@tok@c\endcsname{\let\PY@it=\textit\def\PY@tc##1{\textcolor[rgb]{0.25,0.50,0.50}{##1}}}
\expandafter\def\csname PY@tok@cp\endcsname{\def\PY@tc##1{\textcolor[rgb]{0.74,0.48,0.00}{##1}}}
\expandafter\def\csname PY@tok@k\endcsname{\let\PY@bf=\textbf\def\PY@tc##1{\textcolor[rgb]{0.00,0.50,0.00}{##1}}}
\expandafter\def\csname PY@tok@kp\endcsname{\def\PY@tc##1{\textcolor[rgb]{0.00,0.50,0.00}{##1}}}
\expandafter\def\csname PY@tok@kt\endcsname{\def\PY@tc##1{\textcolor[rgb]{0.69,0.00,0.25}{##1}}}
\expandafter\def\csname PY@tok@o\endcsname{\def\PY@tc##1{\textcolor[rgb]{0.40,0.40,0.40}{##1}}}
\expandafter\def\csname PY@tok@ow\endcsname{\let\PY@bf=\textbf\def\PY@tc##1{\textcolor[rgb]{0.67,0.13,1.00}{##1}}}
\expandafter\def\csname PY@tok@nb\endcsname{\def\PY@tc##1{\textcolor[rgb]{0.00,0.50,0.00}{##1}}}
\expandafter\def\csname PY@tok@nf\endcsname{\def\PY@tc##1{\textcolor[rgb]{0.00,0.00,1.00}{##1}}}
\expandafter\def\csname PY@tok@nc\endcsname{\let\PY@bf=\textbf\def\PY@tc##1{\textcolor[rgb]{0.00,0.00,1.00}{##1}}}
\expandafter\def\csname PY@tok@nn\endcsname{\let\PY@bf=\textbf\def\PY@tc##1{\textcolor[rgb]{0.00,0.00,1.00}{##1}}}
\expandafter\def\csname PY@tok@ne\endcsname{\let\PY@bf=\textbf\def\PY@tc##1{\textcolor[rgb]{0.82,0.25,0.23}{##1}}}
\expandafter\def\csname PY@tok@nv\endcsname{\def\PY@tc##1{\textcolor[rgb]{0.10,0.09,0.49}{##1}}}
\expandafter\def\csname PY@tok@no\endcsname{\def\PY@tc##1{\textcolor[rgb]{0.53,0.00,0.00}{##1}}}
\expandafter\def\csname PY@tok@nl\endcsname{\def\PY@tc##1{\textcolor[rgb]{0.63,0.63,0.00}{##1}}}
\expandafter\def\csname PY@tok@ni\endcsname{\let\PY@bf=\textbf\def\PY@tc##1{\textcolor[rgb]{0.60,0.60,0.60}{##1}}}
\expandafter\def\csname PY@tok@na\endcsname{\def\PY@tc##1{\textcolor[rgb]{0.49,0.56,0.16}{##1}}}
\expandafter\def\csname PY@tok@nt\endcsname{\let\PY@bf=\textbf\def\PY@tc##1{\textcolor[rgb]{0.00,0.50,0.00}{##1}}}
\expandafter\def\csname PY@tok@nd\endcsname{\def\PY@tc##1{\textcolor[rgb]{0.67,0.13,1.00}{##1}}}
\expandafter\def\csname PY@tok@s\endcsname{\def\PY@tc##1{\textcolor[rgb]{0.73,0.13,0.13}{##1}}}
\expandafter\def\csname PY@tok@sd\endcsname{\let\PY@it=\textit\def\PY@tc##1{\textcolor[rgb]{0.73,0.13,0.13}{##1}}}
\expandafter\def\csname PY@tok@si\endcsname{\let\PY@bf=\textbf\def\PY@tc##1{\textcolor[rgb]{0.73,0.40,0.53}{##1}}}
\expandafter\def\csname PY@tok@se\endcsname{\let\PY@bf=\textbf\def\PY@tc##1{\textcolor[rgb]{0.73,0.40,0.13}{##1}}}
\expandafter\def\csname PY@tok@sr\endcsname{\def\PY@tc##1{\textcolor[rgb]{0.73,0.40,0.53}{##1}}}
\expandafter\def\csname PY@tok@ss\endcsname{\def\PY@tc##1{\textcolor[rgb]{0.10,0.09,0.49}{##1}}}
\expandafter\def\csname PY@tok@sx\endcsname{\def\PY@tc##1{\textcolor[rgb]{0.00,0.50,0.00}{##1}}}
\expandafter\def\csname PY@tok@m\endcsname{\def\PY@tc##1{\textcolor[rgb]{0.40,0.40,0.40}{##1}}}
\expandafter\def\csname PY@tok@gh\endcsname{\let\PY@bf=\textbf\def\PY@tc##1{\textcolor[rgb]{0.00,0.00,0.50}{##1}}}
\expandafter\def\csname PY@tok@gu\endcsname{\let\PY@bf=\textbf\def\PY@tc##1{\textcolor[rgb]{0.50,0.00,0.50}{##1}}}
\expandafter\def\csname PY@tok@gd\endcsname{\def\PY@tc##1{\textcolor[rgb]{0.63,0.00,0.00}{##1}}}
\expandafter\def\csname PY@tok@gi\endcsname{\def\PY@tc##1{\textcolor[rgb]{0.00,0.63,0.00}{##1}}}
\expandafter\def\csname PY@tok@gr\endcsname{\def\PY@tc##1{\textcolor[rgb]{1.00,0.00,0.00}{##1}}}
\expandafter\def\csname PY@tok@ge\endcsname{\let\PY@it=\textit}
\expandafter\def\csname PY@tok@gs\endcsname{\let\PY@bf=\textbf}
\expandafter\def\csname PY@tok@gp\endcsname{\let\PY@bf=\textbf\def\PY@tc##1{\textcolor[rgb]{0.00,0.00,0.50}{##1}}}
\expandafter\def\csname PY@tok@go\endcsname{\def\PY@tc##1{\textcolor[rgb]{0.53,0.53,0.53}{##1}}}
\expandafter\def\csname PY@tok@gt\endcsname{\def\PY@tc##1{\textcolor[rgb]{0.00,0.27,0.87}{##1}}}
\expandafter\def\csname PY@tok@err\endcsname{\def\PY@bc##1{\setlength{\fboxsep}{0pt}\fcolorbox[rgb]{1.00,0.00,0.00}{1,1,1}{\strut ##1}}}
\expandafter\def\csname PY@tok@kc\endcsname{\let\PY@bf=\textbf\def\PY@tc##1{\textcolor[rgb]{0.00,0.50,0.00}{##1}}}
\expandafter\def\csname PY@tok@kd\endcsname{\let\PY@bf=\textbf\def\PY@tc##1{\textcolor[rgb]{0.00,0.50,0.00}{##1}}}
\expandafter\def\csname PY@tok@kn\endcsname{\let\PY@bf=\textbf\def\PY@tc##1{\textcolor[rgb]{0.00,0.50,0.00}{##1}}}
\expandafter\def\csname PY@tok@kr\endcsname{\let\PY@bf=\textbf\def\PY@tc##1{\textcolor[rgb]{0.00,0.50,0.00}{##1}}}
\expandafter\def\csname PY@tok@bp\endcsname{\def\PY@tc##1{\textcolor[rgb]{0.00,0.50,0.00}{##1}}}
\expandafter\def\csname PY@tok@fm\endcsname{\def\PY@tc##1{\textcolor[rgb]{0.00,0.00,1.00}{##1}}}
\expandafter\def\csname PY@tok@vc\endcsname{\def\PY@tc##1{\textcolor[rgb]{0.10,0.09,0.49}{##1}}}
\expandafter\def\csname PY@tok@vg\endcsname{\def\PY@tc##1{\textcolor[rgb]{0.10,0.09,0.49}{##1}}}
\expandafter\def\csname PY@tok@vi\endcsname{\def\PY@tc##1{\textcolor[rgb]{0.10,0.09,0.49}{##1}}}
\expandafter\def\csname PY@tok@vm\endcsname{\def\PY@tc##1{\textcolor[rgb]{0.10,0.09,0.49}{##1}}}
\expandafter\def\csname PY@tok@sa\endcsname{\def\PY@tc##1{\textcolor[rgb]{0.73,0.13,0.13}{##1}}}
\expandafter\def\csname PY@tok@sb\endcsname{\def\PY@tc##1{\textcolor[rgb]{0.73,0.13,0.13}{##1}}}
\expandafter\def\csname PY@tok@sc\endcsname{\def\PY@tc##1{\textcolor[rgb]{0.73,0.13,0.13}{##1}}}
\expandafter\def\csname PY@tok@dl\endcsname{\def\PY@tc##1{\textcolor[rgb]{0.73,0.13,0.13}{##1}}}
\expandafter\def\csname PY@tok@s2\endcsname{\def\PY@tc##1{\textcolor[rgb]{0.73,0.13,0.13}{##1}}}
\expandafter\def\csname PY@tok@sh\endcsname{\def\PY@tc##1{\textcolor[rgb]{0.73,0.13,0.13}{##1}}}
\expandafter\def\csname PY@tok@s1\endcsname{\def\PY@tc##1{\textcolor[rgb]{0.73,0.13,0.13}{##1}}}
\expandafter\def\csname PY@tok@mb\endcsname{\def\PY@tc##1{\textcolor[rgb]{0.40,0.40,0.40}{##1}}}
\expandafter\def\csname PY@tok@mf\endcsname{\def\PY@tc##1{\textcolor[rgb]{0.40,0.40,0.40}{##1}}}
\expandafter\def\csname PY@tok@mh\endcsname{\def\PY@tc##1{\textcolor[rgb]{0.40,0.40,0.40}{##1}}}
\expandafter\def\csname PY@tok@mi\endcsname{\def\PY@tc##1{\textcolor[rgb]{0.40,0.40,0.40}{##1}}}
\expandafter\def\csname PY@tok@il\endcsname{\def\PY@tc##1{\textcolor[rgb]{0.40,0.40,0.40}{##1}}}
\expandafter\def\csname PY@tok@mo\endcsname{\def\PY@tc##1{\textcolor[rgb]{0.40,0.40,0.40}{##1}}}
\expandafter\def\csname PY@tok@ch\endcsname{\let\PY@it=\textit\def\PY@tc##1{\textcolor[rgb]{0.25,0.50,0.50}{##1}}}
\expandafter\def\csname PY@tok@cm\endcsname{\let\PY@it=\textit\def\PY@tc##1{\textcolor[rgb]{0.25,0.50,0.50}{##1}}}
\expandafter\def\csname PY@tok@cpf\endcsname{\let\PY@it=\textit\def\PY@tc##1{\textcolor[rgb]{0.25,0.50,0.50}{##1}}}
\expandafter\def\csname PY@tok@c1\endcsname{\let\PY@it=\textit\def\PY@tc##1{\textcolor[rgb]{0.25,0.50,0.50}{##1}}}
\expandafter\def\csname PY@tok@cs\endcsname{\let\PY@it=\textit\def\PY@tc##1{\textcolor[rgb]{0.25,0.50,0.50}{##1}}}

\def\PYZbs{\char`\\}
\def\PYZus{\char`\_}
\def\PYZob{\char`\{}
\def\PYZcb{\char`\}}
\def\PYZca{\char`\^}
\def\PYZam{\char`\&}
\def\PYZlt{\char`\<}
\def\PYZgt{\char`\>}
\def\PYZsh{\char`\#}
\def\PYZpc{\char`\%}
\def\PYZdl{\char`\$}
\def\PYZhy{\char`\-}
\def\PYZsq{\char`\'}
\def\PYZdq{\char`\"}
\def\PYZti{\char`\~}
% for compatibility with earlier versions
\def\PYZat{@}
\def\PYZlb{[}
\def\PYZrb{]}
\makeatother


    % Exact colors from NB
    \definecolor{incolor}{rgb}{0.0, 0.0, 0.5}
    \definecolor{outcolor}{rgb}{0.545, 0.0, 0.0}



    
    % Prevent overflowing lines due to hard-to-break entities
    \sloppy 
    % Setup hyperref package
    \hypersetup{
      breaklinks=true,  % so long urls are correctly broken across lines
      colorlinks=true,
      urlcolor=urlcolor,
      linkcolor=linkcolor,
      citecolor=citecolor,
      }
    % Slightly bigger margins than the latex defaults
    
    \geometry{verbose,tmargin=1in,bmargin=1in,lmargin=1in,rmargin=1in}
    
    

    \begin{document}
    
    
    \maketitle
    
    

    
    \hypertarget{analyze-ab-test-results}{%
\subsection{Analyze A/B Test Results}\label{analyze-ab-test-results}}

You may either submit your notebook through the workspace here, or you
may work from your local machine and submit through the next page.
Either way assure that your code passes the project
\href{https://review.udacity.com/\#!/projects/37e27304-ad47-4eb0-a1ab-8c12f60e43d0/rubric}{RUBRIC}.
**Please save regularly

This project will assure you have mastered the subjects covered in the
statistics lessons. The hope is to have this project be as comprehensive
of these topics as possible. Good luck!

\hypertarget{table-of-contents}{%
\subsection{Table of Contents}\label{table-of-contents}}

\begin{itemize}
\tightlist
\item
  Section \ref{intro}
\item
  Section \ref{probability}
\item
  Section \ref{ab_test}
\item
  Section \ref{regression}
\end{itemize}

 \#\#\# Introduction

A/B tests are very commonly performed by data analysts and data
scientists. It is important that you get some practice working with the
difficulties of these

For this project, you will be working to understand the results of an
A/B test run by an e-commerce website. Your goal is to work through this
notebook to help the company understand if they should implement the new
page, keep the old page, or perhaps run the experiment longer to make
their decision.

\textbf{As you work through this notebook, follow along in the classroom
and answer the corresponding quiz questions associated with each
question.} The labels for each classroom concept are provided for each
question. This will assure you are on the right track as you work
through the project, and you can feel more confident in your final
submission meeting the criteria. As a final check, assure you meet all
the criteria on the
\href{https://review.udacity.com/\#!/projects/37e27304-ad47-4eb0-a1ab-8c12f60e43d0/rubric}{RUBRIC}.

 \#\#\#\# Part I - Probability

To get started, let's import our libraries.

    \begin{Verbatim}[commandchars=\\\{\}]
{\color{incolor}In [{\color{incolor}1}]:} \PY{k+kn}{import} \PY{n+nn}{pandas} \PY{k}{as} \PY{n+nn}{pd}
        \PY{k+kn}{import} \PY{n+nn}{numpy} \PY{k}{as} \PY{n+nn}{np}
        \PY{k+kn}{import} \PY{n+nn}{random}
        \PY{k+kn}{import} \PY{n+nn}{matplotlib}\PY{n+nn}{.}\PY{n+nn}{pyplot} \PY{k}{as} \PY{n+nn}{plt}
        \PY{o}{\PYZpc{}}\PY{k}{matplotlib} inline
        \PY{c+c1}{\PYZsh{}We are setting the seed to assure you get the same answers on quizzes as we set up}
        \PY{n}{random}\PY{o}{.}\PY{n}{seed}\PY{p}{(}\PY{l+m+mi}{42}\PY{p}{)}
\end{Verbatim}


    \texttt{1.} Now, read in the \texttt{ab\_data.csv} data. Store it in
\texttt{df}. \textbf{Use your dataframe to answer the questions in Quiz
1 of the classroom.}

\begin{enumerate}
\def\labelenumi{\alph{enumi}.}
\tightlist
\item
  Read in the dataset and take a look at the top few rows here:
\end{enumerate}

    \begin{Verbatim}[commandchars=\\\{\}]
{\color{incolor}In [{\color{incolor}2}]:} \PY{c+c1}{\PYZsh{} Reading the data file ab\PYZus{}data.csv}
        \PY{n}{df} \PY{o}{=} \PY{n}{pd}\PY{o}{.}\PY{n}{read\PYZus{}csv}\PY{p}{(}\PY{l+s+s1}{\PYZsq{}}\PY{l+s+s1}{ab\PYZus{}data.csv}\PY{l+s+s1}{\PYZsq{}}\PY{p}{)}
        \PY{n}{df}\PY{o}{.}\PY{n}{head}\PY{p}{(}\PY{p}{)}
\end{Verbatim}


\begin{Verbatim}[commandchars=\\\{\}]
{\color{outcolor}Out[{\color{outcolor}2}]:}    user\_id                   timestamp      group landing\_page  converted
        0   851104  2017-01-21 22:11:48.556739    control     old\_page          0
        1   804228  2017-01-12 08:01:45.159739    control     old\_page          0
        2   661590  2017-01-11 16:55:06.154213  treatment     new\_page          0
        3   853541  2017-01-08 18:28:03.143765  treatment     new\_page          0
        4   864975  2017-01-21 01:52:26.210827    control     old\_page          1
\end{Verbatim}
            
    \begin{enumerate}
\def\labelenumi{\alph{enumi}.}
\setcounter{enumi}{1}
\tightlist
\item
  Use the below cell to find the number of rows in the dataset.
\end{enumerate}

    \begin{Verbatim}[commandchars=\\\{\}]
{\color{incolor}In [{\color{incolor}3}]:} \PY{n}{rows} \PY{o}{=} \PY{n}{df}\PY{o}{.}\PY{n}{shape}
        \PY{n}{rows}\PY{p}{[}\PY{l+m+mi}{0}\PY{p}{]}
\end{Verbatim}


\begin{Verbatim}[commandchars=\\\{\}]
{\color{outcolor}Out[{\color{outcolor}3}]:} 294478
\end{Verbatim}
            
    \begin{enumerate}
\def\labelenumi{\alph{enumi}.}
\setcounter{enumi}{2}
\tightlist
\item
  The number of unique users in the dataset.
\end{enumerate}

    \begin{Verbatim}[commandchars=\\\{\}]
{\color{incolor}In [{\color{incolor}4}]:} \PY{n}{number\PYZus{}of\PYZus{}unique\PYZus{}users} \PY{o}{=} \PY{n}{df}\PY{o}{.}\PY{n}{nunique}\PY{p}{(}\PY{p}{)}
        \PY{n}{number\PYZus{}of\PYZus{}unique\PYZus{}users}\PY{p}{[}\PY{l+m+mi}{0}\PY{p}{]}
\end{Verbatim}


\begin{Verbatim}[commandchars=\\\{\}]
{\color{outcolor}Out[{\color{outcolor}4}]:} 290584
\end{Verbatim}
            
    \begin{enumerate}
\def\labelenumi{\alph{enumi}.}
\setcounter{enumi}{3}
\tightlist
\item
  The proportion of users converted.
\end{enumerate}

    \begin{Verbatim}[commandchars=\\\{\}]
{\color{incolor}In [{\color{incolor}5}]:} \PY{c+c1}{\PYZsh{} To calculate the proportion of users converted take the mean}
        \PY{n}{proportion\PYZus{}converted} \PY{o}{=} \PY{n}{df}\PY{p}{[}\PY{l+s+s1}{\PYZsq{}}\PY{l+s+s1}{converted}\PY{l+s+s1}{\PYZsq{}}\PY{p}{]}\PY{o}{.}\PY{n}{mean}\PY{p}{(}\PY{p}{)}
        \PY{c+c1}{\PYZsh{} in percentage}
        \PY{n}{proportion\PYZus{}converted}\PY{o}{*}\PY{l+m+mi}{100}
\end{Verbatim}


\begin{Verbatim}[commandchars=\\\{\}]
{\color{outcolor}Out[{\color{outcolor}5}]:} 11.965919355605511
\end{Verbatim}
            
    \begin{enumerate}
\def\labelenumi{\alph{enumi}.}
\setcounter{enumi}{4}
\tightlist
\item
  The number of times the \texttt{new\_page} and \texttt{treatment}
  don't line up.
\end{enumerate}

    \begin{Verbatim}[commandchars=\\\{\}]
{\color{incolor}In [{\color{incolor}6}]:} \PY{c+c1}{\PYZsh{} Initally calculate when treatment group lands on old page}
        \PY{n}{mismatch1} \PY{o}{=} \PY{n+nb}{len}\PY{p}{(}\PY{n}{df}\PY{o}{.}\PY{n}{query}\PY{p}{(}\PY{l+s+s2}{\PYZdq{}}\PY{l+s+s2}{group == }\PY{l+s+s2}{\PYZsq{}}\PY{l+s+s2}{treatment}\PY{l+s+s2}{\PYZsq{}}\PY{l+s+s2}{ and landing\PYZus{}page == }\PY{l+s+s2}{\PYZsq{}}\PY{l+s+s2}{old\PYZus{}page}\PY{l+s+s2}{\PYZsq{}}\PY{l+s+s2}{\PYZdq{}}\PY{p}{)}\PY{p}{)}
        \PY{c+c1}{\PYZsh{} Now calculate when control group lands on new page}
        \PY{n}{mismatch2} \PY{o}{=} \PY{n+nb}{len}\PY{p}{(}\PY{n}{df}\PY{o}{.}\PY{n}{query}\PY{p}{(}\PY{l+s+s2}{\PYZdq{}}\PY{l+s+s2}{group == }\PY{l+s+s2}{\PYZsq{}}\PY{l+s+s2}{control}\PY{l+s+s2}{\PYZsq{}}\PY{l+s+s2}{ and landing\PYZus{}page == }\PY{l+s+s2}{\PYZsq{}}\PY{l+s+s2}{new\PYZus{}page}\PY{l+s+s2}{\PYZsq{}}\PY{l+s+s2}{\PYZdq{}}\PY{p}{)}\PY{p}{)}
        \PY{c+c1}{\PYZsh{} No of times new\PYZus{}page and treatment dont line up}
        \PY{n}{mismatch1} \PY{o}{+} \PY{n}{mismatch2}
\end{Verbatim}


\begin{Verbatim}[commandchars=\\\{\}]
{\color{outcolor}Out[{\color{outcolor}6}]:} 3893
\end{Verbatim}
            
    \begin{enumerate}
\def\labelenumi{\alph{enumi}.}
\setcounter{enumi}{5}
\tightlist
\item
  Do any of the rows have missing values?
\end{enumerate}

    \begin{Verbatim}[commandchars=\\\{\}]
{\color{incolor}In [{\color{incolor}7}]:} \PY{c+c1}{\PYZsh{} To check for the no of missing values in the data set}
        \PY{n}{df}\PY{o}{.}\PY{n}{info}\PY{p}{(}\PY{p}{)}
        \PY{c+c1}{\PYZsh{} All the columns have the same number of rows so there are no missing values}
\end{Verbatim}


    \begin{Verbatim}[commandchars=\\\{\}]
<class 'pandas.core.frame.DataFrame'>
RangeIndex: 294478 entries, 0 to 294477
Data columns (total 5 columns):
user\_id         294478 non-null int64
timestamp       294478 non-null object
group           294478 non-null object
landing\_page    294478 non-null object
converted       294478 non-null int64
dtypes: int64(2), object(3)
memory usage: 11.2+ MB

    \end{Verbatim}

    \texttt{2.} For the rows where \textbf{treatment} is not aligned with
\textbf{new\_page} or \textbf{control} is not aligned with
\textbf{old\_page}, we cannot be sure if this row truly received the new
or old page. Use \textbf{Quiz 2} in the classroom to provide how we
should handle these rows.

\begin{enumerate}
\def\labelenumi{\alph{enumi}.}
\tightlist
\item
  Now use the answer to the quiz to create a new dataset that meets the
  specifications from the quiz. Store your new dataframe in
  \textbf{df2}.
\end{enumerate}

    \begin{Verbatim}[commandchars=\\\{\}]
{\color{incolor}In [{\color{incolor}8}]:} \PY{c+c1}{\PYZsh{} Delete the rows where treatment is not aligned with new\PYZus{}page}
        \PY{n}{df}\PY{o}{.}\PY{n}{drop}\PY{p}{(}\PY{n}{df}\PY{o}{.}\PY{n}{query}\PY{p}{(}\PY{l+s+s2}{\PYZdq{}}\PY{l+s+s2}{group == }\PY{l+s+s2}{\PYZsq{}}\PY{l+s+s2}{treatment}\PY{l+s+s2}{\PYZsq{}}\PY{l+s+s2}{ and landing\PYZus{}page == }\PY{l+s+s2}{\PYZsq{}}\PY{l+s+s2}{old\PYZus{}page}\PY{l+s+s2}{\PYZsq{}}\PY{l+s+s2}{\PYZdq{}}\PY{p}{)}\PY{o}{.}\PY{n}{index}\PY{p}{,} \PY{n}{inplace}\PY{o}{=}\PY{k+kc}{True}\PY{p}{)}
        \PY{c+c1}{\PYZsh{} Delete the rows where control is not aligned with old page}
        \PY{n}{df}\PY{o}{.}\PY{n}{drop}\PY{p}{(}\PY{n}{df}\PY{o}{.}\PY{n}{query}\PY{p}{(}\PY{l+s+s2}{\PYZdq{}}\PY{l+s+s2}{group == }\PY{l+s+s2}{\PYZsq{}}\PY{l+s+s2}{control}\PY{l+s+s2}{\PYZsq{}}\PY{l+s+s2}{ and landing\PYZus{}page == }\PY{l+s+s2}{\PYZsq{}}\PY{l+s+s2}{new\PYZus{}page}\PY{l+s+s2}{\PYZsq{}}\PY{l+s+s2}{\PYZdq{}}\PY{p}{)}\PY{o}{.}\PY{n}{index}\PY{p}{,} \PY{n}{inplace}\PY{o}{=}\PY{k+kc}{True}\PY{p}{)}
\end{Verbatim}


    \begin{Verbatim}[commandchars=\\\{\}]
{\color{incolor}In [{\color{incolor}9}]:} \PY{c+c1}{\PYZsh{} Find if there are any missing values}
        \PY{n}{df}\PY{o}{.}\PY{n}{info}\PY{p}{(}\PY{p}{)}
        \PY{c+c1}{\PYZsh{} There are no missing values in any of the rows. This is  the updated data frame}
\end{Verbatim}


    \begin{Verbatim}[commandchars=\\\{\}]
<class 'pandas.core.frame.DataFrame'>
Int64Index: 290585 entries, 0 to 294477
Data columns (total 5 columns):
user\_id         290585 non-null int64
timestamp       290585 non-null object
group           290585 non-null object
landing\_page    290585 non-null object
converted       290585 non-null int64
dtypes: int64(2), object(3)
memory usage: 13.3+ MB

    \end{Verbatim}

    \begin{Verbatim}[commandchars=\\\{\}]
{\color{incolor}In [{\color{incolor}10}]:} \PY{c+c1}{\PYZsh{} Save the new Data set in a new .csv file and then create a new data set df2}
         \PY{n}{df}\PY{o}{.}\PY{n}{to\PYZus{}csv}\PY{p}{(}\PY{l+s+s1}{\PYZsq{}}\PY{l+s+s1}{ab\PYZus{}data\PYZus{}new.csv}\PY{l+s+s1}{\PYZsq{}}\PY{p}{,} \PY{n}{index}\PY{o}{=}\PY{k+kc}{False}\PY{p}{)}
\end{Verbatim}


    \begin{Verbatim}[commandchars=\\\{\}]
{\color{incolor}In [{\color{incolor}11}]:} \PY{c+c1}{\PYZsh{} Load the new data set into the new data frame df2}
         \PY{n}{df2} \PY{o}{=} \PY{n}{pd}\PY{o}{.}\PY{n}{read\PYZus{}csv}\PY{p}{(}\PY{l+s+s1}{\PYZsq{}}\PY{l+s+s1}{ab\PYZus{}data\PYZus{}new.csv}\PY{l+s+s1}{\PYZsq{}}\PY{p}{)}
         \PY{n}{df2}\PY{o}{.}\PY{n}{head}\PY{p}{(}\PY{p}{)}
\end{Verbatim}


\begin{Verbatim}[commandchars=\\\{\}]
{\color{outcolor}Out[{\color{outcolor}11}]:}    user\_id                   timestamp      group landing\_page  converted
         0   851104  2017-01-21 22:11:48.556739    control     old\_page          0
         1   804228  2017-01-12 08:01:45.159739    control     old\_page          0
         2   661590  2017-01-11 16:55:06.154213  treatment     new\_page          0
         3   853541  2017-01-08 18:28:03.143765  treatment     new\_page          0
         4   864975  2017-01-21 01:52:26.210827    control     old\_page          1
\end{Verbatim}
            
    \begin{Verbatim}[commandchars=\\\{\}]
{\color{incolor}In [{\color{incolor}12}]:} \PY{c+c1}{\PYZsh{} Double Check all of the correct rows were removed \PYZhy{} this should be 0}
         \PY{n}{df2}\PY{p}{[}\PY{p}{(}\PY{p}{(}\PY{n}{df2}\PY{p}{[}\PY{l+s+s1}{\PYZsq{}}\PY{l+s+s1}{group}\PY{l+s+s1}{\PYZsq{}}\PY{p}{]} \PY{o}{==} \PY{l+s+s1}{\PYZsq{}}\PY{l+s+s1}{treatment}\PY{l+s+s1}{\PYZsq{}}\PY{p}{)} \PY{o}{==} \PY{p}{(}\PY{n}{df2}\PY{p}{[}\PY{l+s+s1}{\PYZsq{}}\PY{l+s+s1}{landing\PYZus{}page}\PY{l+s+s1}{\PYZsq{}}\PY{p}{]} \PY{o}{==} \PY{l+s+s1}{\PYZsq{}}\PY{l+s+s1}{new\PYZus{}page}\PY{l+s+s1}{\PYZsq{}}\PY{p}{)}\PY{p}{)} \PY{o}{==} \PY{k+kc}{False}\PY{p}{]}\PY{o}{.}\PY{n}{shape}\PY{p}{[}\PY{l+m+mi}{0}\PY{p}{]}
\end{Verbatim}


\begin{Verbatim}[commandchars=\\\{\}]
{\color{outcolor}Out[{\color{outcolor}12}]:} 0
\end{Verbatim}
            
    \texttt{3.} Use \textbf{df2} and the cells below to answer questions for
\textbf{Quiz3} in the classroom.

    \begin{enumerate}
\def\labelenumi{\alph{enumi}.}
\tightlist
\item
  How many unique \textbf{user\_id}s are in \textbf{df2}?
\end{enumerate}

    \begin{Verbatim}[commandchars=\\\{\}]
{\color{incolor}In [{\color{incolor}13}]:} \PY{n}{new\PYZus{}unique\PYZus{}users} \PY{o}{=} \PY{n}{df2}\PY{o}{.}\PY{n}{nunique}\PY{p}{(}\PY{p}{)}
         \PY{n}{new\PYZus{}unique\PYZus{}users}\PY{p}{[}\PY{l+m+mi}{0}\PY{p}{]}
\end{Verbatim}


\begin{Verbatim}[commandchars=\\\{\}]
{\color{outcolor}Out[{\color{outcolor}13}]:} 290584
\end{Verbatim}
            
    \begin{enumerate}
\def\labelenumi{\alph{enumi}.}
\setcounter{enumi}{1}
\tightlist
\item
  There is one \textbf{user\_id} repeated in \textbf{df2}. What is it?
\end{enumerate}

    \begin{Verbatim}[commandchars=\\\{\}]
{\color{incolor}In [{\color{incolor}14}]:} \PY{n}{df2}\PY{p}{[}\PY{n}{df2}\PY{o}{.}\PY{n}{duplicated}\PY{p}{(}\PY{p}{[}\PY{l+s+s1}{\PYZsq{}}\PY{l+s+s1}{user\PYZus{}id}\PY{l+s+s1}{\PYZsq{}}\PY{p}{]}\PY{p}{,} \PY{n}{keep}\PY{o}{=}\PY{k+kc}{False}\PY{p}{)}\PY{p}{]}\PY{p}{[}\PY{l+s+s1}{\PYZsq{}}\PY{l+s+s1}{user\PYZus{}id}\PY{l+s+s1}{\PYZsq{}}\PY{p}{]}
\end{Verbatim}


\begin{Verbatim}[commandchars=\\\{\}]
{\color{outcolor}Out[{\color{outcolor}14}]:} 1876    773192
         2862    773192
         Name: user\_id, dtype: int64
\end{Verbatim}
            
    \begin{enumerate}
\def\labelenumi{\alph{enumi}.}
\setcounter{enumi}{2}
\tightlist
\item
  What is the row information for the repeat \textbf{user\_id}?
\end{enumerate}

    \begin{Verbatim}[commandchars=\\\{\}]
{\color{incolor}In [{\color{incolor}15}]:} \PY{n}{df2}\PY{p}{[}\PY{n}{df2}\PY{o}{.}\PY{n}{duplicated}\PY{p}{(}\PY{p}{[}\PY{l+s+s1}{\PYZsq{}}\PY{l+s+s1}{user\PYZus{}id}\PY{l+s+s1}{\PYZsq{}}\PY{p}{]}\PY{p}{,} \PY{n}{keep}\PY{o}{=}\PY{k+kc}{False}\PY{p}{)}\PY{p}{]}
\end{Verbatim}


\begin{Verbatim}[commandchars=\\\{\}]
{\color{outcolor}Out[{\color{outcolor}15}]:}       user\_id                   timestamp      group landing\_page  converted
         1876   773192  2017-01-09 05:37:58.781806  treatment     new\_page          0
         2862   773192  2017-01-14 02:55:59.590927  treatment     new\_page          0
\end{Verbatim}
            
    \begin{enumerate}
\def\labelenumi{\alph{enumi}.}
\setcounter{enumi}{3}
\tightlist
\item
  Remove \textbf{one} of the rows with a duplicate \textbf{user\_id},
  but keep your dataframe as \textbf{df2}.
\end{enumerate}

    \begin{Verbatim}[commandchars=\\\{\}]
{\color{incolor}In [{\color{incolor}16}]:} \PY{c+c1}{\PYZsh{} Remove one of the duplicate user\PYZus{}id from the data set}
         \PY{n}{df2}\PY{o}{.}\PY{n}{drop}\PY{p}{(}\PY{n}{df2}\PY{o}{.}\PY{n}{duplicated}\PY{p}{(}\PY{p}{)}\PY{p}{)}
         \PY{n+nb}{sum}\PY{p}{(}\PY{n}{df2}\PY{o}{.}\PY{n}{duplicated}\PY{p}{(}\PY{p}{)}\PY{p}{)}   \PY{c+c1}{\PYZsh{} Check if the duplicate row has been deleted}
\end{Verbatim}


\begin{Verbatim}[commandchars=\\\{\}]
{\color{outcolor}Out[{\color{outcolor}16}]:} 0
\end{Verbatim}
            
    \texttt{4.} Use \textbf{df2} in the below cells to answer the quiz
questions related to \textbf{Quiz 4} in the classroom.

\begin{enumerate}
\def\labelenumi{\alph{enumi}.}
\tightlist
\item
  What is the probability of an individual converting regardless of the
  page they receive?
\end{enumerate}

    \begin{Verbatim}[commandchars=\\\{\}]
{\color{incolor}In [{\color{incolor}17}]:} \PY{c+c1}{\PYZsh{} To calculate the propability of a series we can take the mean if that because }
         \PY{c+c1}{\PYZsh{} the values are either 0 or 1}
         \PY{n}{df2}\PY{p}{[}\PY{l+s+s1}{\PYZsq{}}\PY{l+s+s1}{converted}\PY{l+s+s1}{\PYZsq{}}\PY{p}{]}\PY{o}{.}\PY{n}{mean}\PY{p}{(}\PY{p}{)}
\end{Verbatim}


\begin{Verbatim}[commandchars=\\\{\}]
{\color{outcolor}Out[{\color{outcolor}17}]:} 0.11959667567149027
\end{Verbatim}
            
    \begin{enumerate}
\def\labelenumi{\alph{enumi}.}
\setcounter{enumi}{1}
\tightlist
\item
  Given that an individual was in the \texttt{control} group, what is
  the probability they converted?
\end{enumerate}

An individual in the \texttt{control} group has a probability of
0.120386 being converted

    \begin{Verbatim}[commandchars=\\\{\}]
{\color{incolor}In [{\color{incolor}18}]:} \PY{c+c1}{\PYZsh{} Use groupby function to divide it into two groups and then use the mean}
         \PY{c+c1}{\PYZsh{} function to find the probabilty of each group}
         \PY{n}{df\PYZus{}new} \PY{o}{=} \PY{n}{df2}\PY{o}{.}\PY{n}{groupby}\PY{p}{(}\PY{l+s+s1}{\PYZsq{}}\PY{l+s+s1}{group}\PY{l+s+s1}{\PYZsq{}}\PY{p}{)}
         \PY{n}{df\PYZus{}new}\PY{o}{.}\PY{n}{head}\PY{p}{(}\PY{p}{)}
         \PY{n}{df\PYZus{}new} \PY{o}{=} \PY{n}{df\PYZus{}new}\PY{p}{[}\PY{l+s+s1}{\PYZsq{}}\PY{l+s+s1}{converted}\PY{l+s+s1}{\PYZsq{}}\PY{p}{]}\PY{o}{.}\PY{n}{mean}\PY{p}{(}\PY{p}{)}
         \PY{n}{df\PYZus{}new}
\end{Verbatim}


\begin{Verbatim}[commandchars=\\\{\}]
{\color{outcolor}Out[{\color{outcolor}18}]:} group
         control      0.120386
         treatment    0.118807
         Name: converted, dtype: float64
\end{Verbatim}
            
    \begin{enumerate}
\def\labelenumi{\alph{enumi}.}
\setcounter{enumi}{2}
\tightlist
\item
  Given that an individual was in the \texttt{treatment} group, what is
  the probability they converted?
\end{enumerate}

    An individual in the \texttt{treatment} group has a probaility 0.118807
being converted.

    \begin{enumerate}
\def\labelenumi{\alph{enumi}.}
\setcounter{enumi}{3}
\tightlist
\item
  What is the probability that an individual received the new page?
\end{enumerate}

    \begin{Verbatim}[commandchars=\\\{\}]
{\color{incolor}In [{\color{incolor}19}]:} \PY{c+c1}{\PYZsh{} The individual who will receive the new page should be in the treatment group}
         \PY{c+c1}{\PYZsh{} As per the previous data treatment group and new page should be aligned.}
         \PY{c+c1}{\PYZsh{} first find the number of individuals in treatment group in whole data set}
         \PY{n}{no\PYZus{}of\PYZus{}individuals\PYZus{}treatment} \PY{o}{=} \PY{n+nb}{len}\PY{p}{(}\PY{n}{df2}\PY{o}{.}\PY{n}{query}\PY{p}{(}\PY{l+s+s2}{\PYZdq{}}\PY{l+s+s2}{group == }\PY{l+s+s2}{\PYZsq{}}\PY{l+s+s2}{treatment}\PY{l+s+s2}{\PYZsq{}}\PY{l+s+s2}{\PYZdq{}}\PY{p}{)}\PY{p}{)}
         \PY{c+c1}{\PYZsh{} find the total no of inviduals in the whole data set}
         \PY{n}{total\PYZus{}individuals} \PY{o}{=} \PY{n}{df2}\PY{o}{.}\PY{n}{shape}\PY{p}{[}\PY{l+m+mi}{0}\PY{p}{]}
         
         \PY{n}{prob\PYZus{}treatment\PYZus{}group} \PY{o}{=} \PY{n}{no\PYZus{}of\PYZus{}individuals\PYZus{}treatment}\PY{o}{/}\PY{n}{total\PYZus{}individuals}
         \PY{n}{prob\PYZus{}treatment\PYZus{}group}
\end{Verbatim}


\begin{Verbatim}[commandchars=\\\{\}]
{\color{outcolor}Out[{\color{outcolor}19}]:} 0.5000636646764286
\end{Verbatim}
            
    \begin{enumerate}
\def\labelenumi{\alph{enumi}.}
\setcounter{enumi}{4}
\tightlist
\item
  Use the results in the previous two portions of this question to
  suggest if you think there is evidence that one page leads to more
  conversions? Write your response below.
\end{enumerate}

    \textbf{Your answer goes here.}

As per the previous two questions we find the probability of indiviuals
in treatment group and in control group being converted 1. The
probability of individuals in the control group being converted is
0.1203 2. The probability of individuals in the treatment group being
converted is 0.1188 The difference in both these results is very minor
and doesnot provide enough evidence that one page leads to more
conversions.

     \#\#\# Part II - A/B Test

Notice that because of the time stamp associated with each event, you
could technically run a hypothesis test continuously as each observation
was observed.

However, then the hard question is do you stop as soon as one page is
considered significantly better than another or does it need to happen
consistently for a certain amount of time? How long do you run to render
a decision that neither page is better than another?

These questions are the difficult parts associated with A/B tests in
general.

\texttt{1.} For now, consider you need to make the decision just based
on all the data provided. If you want to assume that the old page is
better unless the new page proves to be definitely better at a Type I
error rate of 5\%, what should your null and alternative hypotheses be?
You can state your hypothesis in terms of words or in terms of
\textbf{\(p_{old}\)} and \textbf{\(p_{new}\)}, which are the converted
rates for the old and new pages.

    \textbf{Put your answer here.}

Answer: Alpha = 0.05 because the Type I error rate is 5\% As per the
considerations the null hypothesis will be H0: \(p_{old}\)
\textgreater{}= \(p_{new}\)

H1: \(p_{new}\) \textgreater{} \(p_{old}\)

    \texttt{2.} Assume under the null hypothesis, \(p_{new}\) and
\(p_{old}\) both have ``true'' success rates equal to the
\textbf{converted} success rate regardless of page - that is \(p_{new}\)
and \(p_{old}\) are equal. Furthermore, assume they are equal to the
\textbf{converted} rate in \textbf{ab\_data.csv} regardless of the page.

Use a sample size for each page equal to the ones in
\textbf{ab\_data.csv}.

Perform the sampling distribution for the difference in
\textbf{converted} between the two pages over 10,000 iterations of
calculating an estimate from the null.

Use the cells below to provide the necessary parts of this simulation.
If this doesn't make complete sense right now, don't worry - you are
going to work through the problems below to complete this problem. You
can use \textbf{Quiz 5} in the classroom to make sure you are on the
right track.

    \begin{enumerate}
\def\labelenumi{\alph{enumi}.}
\tightlist
\item
  What is the \textbf{convert rate} for \(p_{new}\) under the null?
\end{enumerate}

    \begin{Verbatim}[commandchars=\\\{\}]
{\color{incolor}In [{\color{incolor}20}]:} \PY{n}{p\PYZus{}new} \PY{o}{=} \PY{n}{df2}\PY{p}{[}\PY{l+s+s1}{\PYZsq{}}\PY{l+s+s1}{converted}\PY{l+s+s1}{\PYZsq{}}\PY{p}{]}\PY{o}{.}\PY{n}{mean}\PY{p}{(}\PY{p}{)}
         \PY{n}{p\PYZus{}new}
\end{Verbatim}


\begin{Verbatim}[commandchars=\\\{\}]
{\color{outcolor}Out[{\color{outcolor}20}]:} 0.11959667567149027
\end{Verbatim}
            
    \begin{enumerate}
\def\labelenumi{\alph{enumi}.}
\setcounter{enumi}{1}
\tightlist
\item
  What is the \textbf{convert rate} for \(p_{old}\) under the null? 
\end{enumerate}

    \begin{Verbatim}[commandchars=\\\{\}]
{\color{incolor}In [{\color{incolor}21}]:} \PY{n}{p\PYZus{}old} \PY{o}{=} \PY{n}{df2}\PY{p}{[}\PY{l+s+s1}{\PYZsq{}}\PY{l+s+s1}{converted}\PY{l+s+s1}{\PYZsq{}}\PY{p}{]}\PY{o}{.}\PY{n}{mean}\PY{p}{(}\PY{p}{)}
         \PY{n}{p\PYZus{}old}
\end{Verbatim}


\begin{Verbatim}[commandchars=\\\{\}]
{\color{outcolor}Out[{\color{outcolor}21}]:} 0.11959667567149027
\end{Verbatim}
            
    \begin{enumerate}
\def\labelenumi{\alph{enumi}.}
\setcounter{enumi}{2}
\tightlist
\item
  What is \(n_{new}\)?
\end{enumerate}

    \begin{Verbatim}[commandchars=\\\{\}]
{\color{incolor}In [{\color{incolor}22}]:} \PY{n}{n\PYZus{}new} \PY{o}{=} \PY{n+nb}{len}\PY{p}{(}\PY{n}{df2}\PY{o}{.}\PY{n}{query}\PY{p}{(}\PY{l+s+s2}{\PYZdq{}}\PY{l+s+s2}{group == }\PY{l+s+s2}{\PYZsq{}}\PY{l+s+s2}{treatment}\PY{l+s+s2}{\PYZsq{}}\PY{l+s+s2}{\PYZdq{}}\PY{p}{)}\PY{p}{)}
         \PY{n}{n\PYZus{}new}
\end{Verbatim}


\begin{Verbatim}[commandchars=\\\{\}]
{\color{outcolor}Out[{\color{outcolor}22}]:} 145311
\end{Verbatim}
            
    \begin{enumerate}
\def\labelenumi{\alph{enumi}.}
\setcounter{enumi}{3}
\tightlist
\item
  What is \(n_{old}\)?
\end{enumerate}

    \begin{Verbatim}[commandchars=\\\{\}]
{\color{incolor}In [{\color{incolor}23}]:} \PY{n}{n\PYZus{}old} \PY{o}{=} \PY{n+nb}{len}\PY{p}{(}\PY{n}{df2}\PY{o}{.}\PY{n}{query}\PY{p}{(}\PY{l+s+s2}{\PYZdq{}}\PY{l+s+s2}{group == }\PY{l+s+s2}{\PYZsq{}}\PY{l+s+s2}{control}\PY{l+s+s2}{\PYZsq{}}\PY{l+s+s2}{\PYZdq{}}\PY{p}{)}\PY{p}{)}
         \PY{n}{n\PYZus{}old}
\end{Verbatim}


\begin{Verbatim}[commandchars=\\\{\}]
{\color{outcolor}Out[{\color{outcolor}23}]:} 145274
\end{Verbatim}
            
    \begin{enumerate}
\def\labelenumi{\alph{enumi}.}
\setcounter{enumi}{4}
\tightlist
\item
  Simulate \(n_{new}\) transactions with a convert rate of \(p_{new}\)
  under the null. Store these \(n_{new}\) 1's and 0's in
  \textbf{new\_page\_converted}.
\end{enumerate}

    \begin{Verbatim}[commandchars=\\\{\}]
{\color{incolor}In [{\color{incolor}24}]:} \PY{c+c1}{\PYZsh{} To simulate the n\PYZus{}new transactions use np.random.choice}
         \PY{n}{new\PYZus{}page\PYZus{}converted} \PY{o}{=} \PY{n}{np}\PY{o}{.}\PY{n}{random}\PY{o}{.}\PY{n}{choice}\PY{p}{(}\PY{p}{[}\PY{l+m+mi}{1}\PY{p}{,}\PY{l+m+mi}{0}\PY{p}{]}\PY{p}{,} \PY{n}{size} \PY{o}{=} \PY{n}{n\PYZus{}new}\PY{p}{,} \PY{n}{p}\PY{o}{=}\PY{p}{[}\PY{n}{p\PYZus{}new}\PY{p}{,} \PY{p}{(}\PY{l+m+mi}{1}\PY{o}{\PYZhy{}}\PY{n}{p\PYZus{}new}\PY{p}{)}\PY{p}{]}\PY{p}{)}
         \PY{n}{new\PYZus{}page\PYZus{}converted}
\end{Verbatim}


\begin{Verbatim}[commandchars=\\\{\}]
{\color{outcolor}Out[{\color{outcolor}24}]:} array([0, 0, 0, {\ldots}, 0, 1, 1])
\end{Verbatim}
            
    \begin{enumerate}
\def\labelenumi{\alph{enumi}.}
\setcounter{enumi}{5}
\tightlist
\item
  Simulate \(n_{old}\) transactions with a convert rate of \(p_{old}\)
  under the null. Store these \(n_{old}\) 1's and 0's in
  \textbf{old\_page\_converted}.
\end{enumerate}

    \begin{Verbatim}[commandchars=\\\{\}]
{\color{incolor}In [{\color{incolor}25}]:} \PY{n}{old\PYZus{}page\PYZus{}converted} \PY{o}{=} \PY{n}{np}\PY{o}{.}\PY{n}{random}\PY{o}{.}\PY{n}{choice}\PY{p}{(}\PY{p}{[}\PY{l+m+mi}{1}\PY{p}{,}\PY{l+m+mi}{0}\PY{p}{]}\PY{p}{,} \PY{n}{size} \PY{o}{=} \PY{n}{n\PYZus{}old}\PY{p}{,} \PY{n}{p}\PY{o}{=}\PY{p}{[}\PY{n}{p\PYZus{}old}\PY{p}{,} \PY{p}{(}\PY{l+m+mi}{1}\PY{o}{\PYZhy{}}\PY{n}{p\PYZus{}old}\PY{p}{)}\PY{p}{]}\PY{p}{)}
         \PY{n}{old\PYZus{}page\PYZus{}converted}
\end{Verbatim}


\begin{Verbatim}[commandchars=\\\{\}]
{\color{outcolor}Out[{\color{outcolor}25}]:} array([0, 1, 0, {\ldots}, 0, 0, 1])
\end{Verbatim}
            
    \begin{enumerate}
\def\labelenumi{\alph{enumi}.}
\setcounter{enumi}{6}
\tightlist
\item
  Find \(p_{new}\) - \(p_{old}\) for your simulated values from part (e)
  and (f).
\end{enumerate}

    \begin{Verbatim}[commandchars=\\\{\}]
{\color{incolor}In [{\color{incolor}26}]:} \PY{c+c1}{\PYZsh{} To calculate p\PYZus{}new from the simulated values of part e}
         \PY{c+c1}{\PYZsh{}p\PYZus{}new = new\PYZus{}page\PYZus{}converted/n\PYZus{}new}
         \PY{c+c1}{\PYZsh{}p\PYZus{}old = old\PYZus{}page\PYZus{}converted/n\PYZus{}old}
         \PY{c+c1}{\PYZsh{} Here we need to make both the arrays of same size to calculate the difference}
         \PY{c+c1}{\PYZsh{} p\PYZus{}new and p\PYZus{}old}
         \PY{n}{new\PYZus{}page\PYZus{}converted}\PY{o}{.}\PY{n}{shape}\PY{p}{[}\PY{l+m+mi}{0}\PY{p}{]}   \PY{c+c1}{\PYZsh{}145311}
         \PY{n}{old\PYZus{}page\PYZus{}converted}\PY{o}{.}\PY{n}{shape}\PY{p}{[}\PY{l+m+mi}{0}\PY{p}{]}   \PY{c+c1}{\PYZsh{}145274}
         \PY{n}{new\PYZus{}page\PYZus{}converted} \PY{o}{=} \PY{n}{new\PYZus{}page\PYZus{}converted}\PY{p}{[}\PY{p}{:}\PY{l+m+mi}{145274}\PY{p}{]} \PY{c+c1}{\PYZsh{} truncating the remaining values}
         \PY{n}{p\PYZus{}diff} \PY{o}{=} \PY{n}{new\PYZus{}page\PYZus{}converted}\PY{o}{/}\PY{n}{n\PYZus{}new} \PY{o}{\PYZhy{}} \PY{n}{old\PYZus{}page\PYZus{}converted}\PY{o}{/}\PY{n}{n\PYZus{}old}
         \PY{n}{p\PYZus{}diff}
\end{Verbatim}


\begin{Verbatim}[commandchars=\\\{\}]
{\color{outcolor}Out[{\color{outcolor}26}]:} array([  0.00000000e+00,  -6.88354420e-06,   0.00000000e+00, {\ldots},
                  0.00000000e+00,   0.00000000e+00,  -6.88354420e-06])
\end{Verbatim}
            
    \begin{enumerate}
\def\labelenumi{\alph{enumi}.}
\setcounter{enumi}{7}
\tightlist
\item
  Simulate 10,000 \(p_{new}\) - \(p_{old}\) values using this same
  process similarly to the one you calculated in parts \textbf{a.
  through g.} above. Store all 10,000 values in \textbf{p\_diffs}.
\end{enumerate}

    \begin{Verbatim}[commandchars=\\\{\}]
{\color{incolor}In [{\color{incolor}27}]:} \PY{n}{p\PYZus{}diffs} \PY{o}{=} \PY{p}{[}\PY{p}{]}
         
         \PY{k}{for} \PY{n}{\PYZus{}} \PY{o+ow}{in} \PY{n+nb}{range}\PY{p}{(}\PY{l+m+mi}{10000}\PY{p}{)}\PY{p}{:}
             \PY{n}{new\PYZus{}page\PYZus{}converted} \PY{o}{=} \PY{n}{np}\PY{o}{.}\PY{n}{random}\PY{o}{.}\PY{n}{choice}\PY{p}{(}\PY{p}{[}\PY{l+m+mi}{1}\PY{p}{,}\PY{l+m+mi}{0}\PY{p}{]}\PY{p}{,} \PY{n}{size} \PY{o}{=} \PY{n}{n\PYZus{}new}\PY{p}{,} \PY{n}{p}\PY{o}{=}\PY{p}{[}\PY{n}{p\PYZus{}new}\PY{p}{,} \PY{p}{(}\PY{l+m+mi}{1}\PY{o}{\PYZhy{}}\PY{n}{p\PYZus{}new}\PY{p}{)}\PY{p}{]}\PY{p}{)}\PY{o}{.}\PY{n}{mean}\PY{p}{(}\PY{p}{)}
             \PY{n}{old\PYZus{}page\PYZus{}converted} \PY{o}{=} \PY{n}{np}\PY{o}{.}\PY{n}{random}\PY{o}{.}\PY{n}{choice}\PY{p}{(}\PY{p}{[}\PY{l+m+mi}{1}\PY{p}{,}\PY{l+m+mi}{0}\PY{p}{]}\PY{p}{,} \PY{n}{size} \PY{o}{=} \PY{n}{n\PYZus{}old}\PY{p}{,} \PY{n}{p}\PY{o}{=}\PY{p}{[}\PY{n}{p\PYZus{}old}\PY{p}{,} \PY{p}{(}\PY{l+m+mi}{1}\PY{o}{\PYZhy{}}\PY{n}{p\PYZus{}old}\PY{p}{)}\PY{p}{]}\PY{p}{)}\PY{o}{.}\PY{n}{mean}\PY{p}{(}\PY{p}{)}
             \PY{n}{diff} \PY{o}{=} \PY{n}{new\PYZus{}page\PYZus{}converted} \PY{o}{\PYZhy{}} \PY{n}{old\PYZus{}page\PYZus{}converted}
             \PY{n}{p\PYZus{}diffs}\PY{o}{.}\PY{n}{append}\PY{p}{(}\PY{n}{diff}\PY{p}{)}
\end{Verbatim}


    \begin{enumerate}
\def\labelenumi{\roman{enumi}.}
\tightlist
\item
  Plot a histogram of the \textbf{p\_diffs}. Does this plot look like
  what you expected? Use the matching problem in the classroom to assure
  you fully understand what was computed here.
\end{enumerate}

    \begin{Verbatim}[commandchars=\\\{\}]
{\color{incolor}In [{\color{incolor}28}]:} \PY{n}{plt}\PY{o}{.}\PY{n}{hist}\PY{p}{(}\PY{n}{p\PYZus{}diffs}\PY{p}{)}\PY{p}{;}
\end{Verbatim}


    \begin{center}
    \adjustimage{max size={0.9\linewidth}{0.9\paperheight}}{output_59_0.png}
    \end{center}
    { \hspace*{\fill} \\}
    
    \begin{enumerate}
\def\labelenumi{\alph{enumi}.}
\setcounter{enumi}{9}
\tightlist
\item
  What proportion of the \textbf{p\_diffs} are greater than the actual
  difference observed in \textbf{ab\_data.csv}?
\end{enumerate}

    \begin{Verbatim}[commandchars=\\\{\}]
{\color{incolor}In [{\color{incolor}29}]:} \PY{n}{actual\PYZus{}difference} \PY{o}{=} \PY{n}{df\PYZus{}new}\PY{p}{[}\PY{l+m+mi}{1}\PY{p}{]} \PY{o}{\PYZhy{}} \PY{n}{df\PYZus{}new}\PY{p}{[}\PY{l+m+mi}{0}\PY{p}{]}
         \PY{n}{actual\PYZus{}difference}
\end{Verbatim}


\begin{Verbatim}[commandchars=\\\{\}]
{\color{outcolor}Out[{\color{outcolor}29}]:} -0.0015790565976871451
\end{Verbatim}
            
    \begin{Verbatim}[commandchars=\\\{\}]
{\color{incolor}In [{\color{incolor}30}]:} \PY{p}{(}\PY{n}{actual\PYZus{}difference} \PY{o}{\PYZlt{}} \PY{n}{p\PYZus{}diffs}\PY{p}{)}\PY{o}{.}\PY{n}{mean}\PY{p}{(}\PY{p}{)}
\end{Verbatim}


\begin{Verbatim}[commandchars=\\\{\}]
{\color{outcolor}Out[{\color{outcolor}30}]:} 0.90359999999999996
\end{Verbatim}
            
    \begin{enumerate}
\def\labelenumi{\alph{enumi}.}
\setcounter{enumi}{10}
\tightlist
\item
  In words, explain what you just computed in part \textbf{j.}. What is
  this value called in scientific studies? What does this value mean in
  terms of whether or not there is a difference between the new and old
  pages?
\end{enumerate}

    \textbf{Put your answer here.}

Answer: Calculated the p-values which are used in Hypothesis testing. It
is the propability value of observing the statistic (or one more extreme
in favour of alternative) if the null hypothesis is true. As seen in the
previous section we don't find much difference in probabilities between
the new pages and old pages.

    \begin{enumerate}
\def\labelenumi{\alph{enumi}.}
\setcounter{enumi}{11}
\tightlist
\item
  We could also use a built-in to achieve similar results. Though using
  the built-in might be easier to code, the above portions are a
  walkthrough of the ideas that are critical to correctly thinking about
  statistical significance. Fill in the below to calculate the number of
  conversions for each page, as well as the number of individuals who
  received each page. Let \texttt{n\_old} and \texttt{n\_new} refer the
  the number of rows associated with the old page and new pages,
  respectively.
\end{enumerate}

    \begin{Verbatim}[commandchars=\\\{\}]
{\color{incolor}In [{\color{incolor}31}]:} \PY{k+kn}{import} \PY{n+nn}{statsmodels}\PY{n+nn}{.}\PY{n+nn}{api} \PY{k}{as} \PY{n+nn}{sm}
         
         \PY{n}{convert\PYZus{}old} \PY{o}{=} \PY{n+nb}{sum}\PY{p}{(}\PY{n}{df2}\PY{o}{.}\PY{n}{query}\PY{p}{(}\PY{l+s+s2}{\PYZdq{}}\PY{l+s+s2}{group == }\PY{l+s+s2}{\PYZsq{}}\PY{l+s+s2}{control}\PY{l+s+s2}{\PYZsq{}}\PY{l+s+s2}{\PYZdq{}}\PY{p}{)}\PY{p}{[}\PY{l+s+s1}{\PYZsq{}}\PY{l+s+s1}{converted}\PY{l+s+s1}{\PYZsq{}}\PY{p}{]}\PY{p}{)}
         \PY{n}{convert\PYZus{}new} \PY{o}{=} \PY{n+nb}{sum}\PY{p}{(}\PY{n}{df2}\PY{o}{.}\PY{n}{query}\PY{p}{(}\PY{l+s+s2}{\PYZdq{}}\PY{l+s+s2}{group == }\PY{l+s+s2}{\PYZsq{}}\PY{l+s+s2}{treatment}\PY{l+s+s2}{\PYZsq{}}\PY{l+s+s2}{\PYZdq{}}\PY{p}{)}\PY{p}{[}\PY{l+s+s1}{\PYZsq{}}\PY{l+s+s1}{converted}\PY{l+s+s1}{\PYZsq{}}\PY{p}{]}\PY{p}{)}
         \PY{n}{n\PYZus{}old} \PY{o}{=} \PY{n+nb}{len}\PY{p}{(}\PY{n}{df2}\PY{o}{.}\PY{n}{query}\PY{p}{(}\PY{l+s+s2}{\PYZdq{}}\PY{l+s+s2}{group == }\PY{l+s+s2}{\PYZsq{}}\PY{l+s+s2}{control}\PY{l+s+s2}{\PYZsq{}}\PY{l+s+s2}{\PYZdq{}}\PY{p}{)}\PY{p}{)}
         \PY{n}{n\PYZus{}new} \PY{o}{=} \PY{n+nb}{len}\PY{p}{(}\PY{n}{df2}\PY{o}{.}\PY{n}{query}\PY{p}{(}\PY{l+s+s2}{\PYZdq{}}\PY{l+s+s2}{group == }\PY{l+s+s2}{\PYZsq{}}\PY{l+s+s2}{treatment}\PY{l+s+s2}{\PYZsq{}}\PY{l+s+s2}{\PYZdq{}}\PY{p}{)}\PY{p}{)}
\end{Verbatim}


    \begin{Verbatim}[commandchars=\\\{\}]
/opt/conda/lib/python3.6/site-packages/statsmodels/compat/pandas.py:56: FutureWarning: The pandas.core.datetools module is deprecated and will be removed in a future version. Please use the pandas.tseries module instead.
  from pandas.core import datetools

    \end{Verbatim}

    \begin{enumerate}
\def\labelenumi{\alph{enumi}.}
\setcounter{enumi}{12}
\tightlist
\item
  Now use \texttt{stats.proportions\_ztest} to compute your test
  statistic and p-value.
  \href{http://knowledgetack.com/python/statsmodels/proportions_ztest/}{Here}
  is a helpful link on using the built in.
\end{enumerate}

    \begin{Verbatim}[commandchars=\\\{\}]
{\color{incolor}In [{\color{incolor}34}]:} \PY{n}{z\PYZus{}score}\PY{p}{,} \PY{n}{p\PYZus{}value} \PY{o}{=} \PY{n}{sm}\PY{o}{.}\PY{n}{stats}\PY{o}{.}\PY{n}{proportions\PYZus{}ztest}\PY{p}{(}\PY{p}{[}\PY{n}{convert\PYZus{}old}\PY{p}{,} \PY{n}{convert\PYZus{}new}\PY{p}{]}\PY{p}{,} \PY{p}{[}\PY{n}{n\PYZus{}old}\PY{p}{,} \PY{n}{n\PYZus{}new}\PY{p}{]}\PY{p}{,} \PY{n}{alternative}\PY{o}{=}\PY{l+s+s1}{\PYZsq{}}\PY{l+s+s1}{smaller}\PY{l+s+s1}{\PYZsq{}}\PY{p}{)}
         \PY{n}{z\PYZus{}score}\PY{p}{,} \PY{n}{p\PYZus{}value}
\end{Verbatim}


\begin{Verbatim}[commandchars=\\\{\}]
{\color{outcolor}Out[{\color{outcolor}34}]:} (1.3116075339133115, 0.90517370514059103)
\end{Verbatim}
            
    \begin{Verbatim}[commandchars=\\\{\}]
{\color{incolor}In [{\color{incolor}35}]:} \PY{k+kn}{from} \PY{n+nn}{scipy}\PY{n+nn}{.}\PY{n+nn}{stats} \PY{k}{import} \PY{n}{norm}
         
         \PY{n}{norm}\PY{o}{.}\PY{n}{cdf}\PY{p}{(}\PY{n}{z\PYZus{}score}\PY{p}{)}
         \PY{c+c1}{\PYZsh{} 0.90517370514059103 \PYZsh{} Tells us how significant our z\PYZhy{}score is}
         
         \PY{n}{norm}\PY{o}{.}\PY{n}{ppf}\PY{p}{(}\PY{l+m+mi}{1}\PY{o}{\PYZhy{}}\PY{p}{(}\PY{l+m+mf}{0.05}\PY{p}{)}\PY{p}{)}
         \PY{c+c1}{\PYZsh{} 1.959963984540054 \PYZsh{} Tells us what our critical value at 95\PYZpc{} confidence is}
\end{Verbatim}


\begin{Verbatim}[commandchars=\\\{\}]
{\color{outcolor}Out[{\color{outcolor}35}]:} 1.6448536269514722
\end{Verbatim}
            
    \begin{enumerate}
\def\labelenumi{\alph{enumi}.}
\setcounter{enumi}{13}
\tightlist
\item
  What do the z-score and p-value you computed in the previous question
  mean for the conversion rates of the old and new pages? Do they agree
  with the findings in parts \textbf{j.} and \textbf{k.}?
\end{enumerate}

    The z-score and p-value computed in the previous sections convey the
message that we should go for the null hypothesis.

The z-score of 1.31 is less than the critical value of 1.644 so we
accept the null hypothesis.

As per the findings in parts j and k there is a very small difference
between the new pages and old pages. But the old pages are slightly
better

     \#\#\# Part III - A regression approach

\texttt{1.} In this final part, you will see that the result you
acheived in the previous A/B test can also be acheived by performing
regression.

\begin{enumerate}
\def\labelenumi{\alph{enumi}.}
\tightlist
\item
  Since each row is either a conversion or no conversion, what type of
  regression should you be performing in this case?
\end{enumerate}

    Since it only has binary results we use Logistic Regression

    \begin{enumerate}
\def\labelenumi{\alph{enumi}.}
\setcounter{enumi}{1}
\tightlist
\item
  The goal is to use \textbf{statsmodels} to fit the regression model
  you specified in part \textbf{a.} to see if there is a significant
  difference in conversion based on which page a customer receives.
  However, you first need to create a colun for the intercept, and
  create a dummy variable column for which page each user received. Add
  an \textbf{intercept} column, as well as an \textbf{ab\_page} column,
  which is 1 when an individual receives the \textbf{treatment} and 0 if
  \textbf{control}.
\end{enumerate}

    \begin{Verbatim}[commandchars=\\\{\}]
{\color{incolor}In [{\color{incolor}36}]:} \PY{n}{df2}\PY{p}{[}\PY{l+s+s1}{\PYZsq{}}\PY{l+s+s1}{intercept}\PY{l+s+s1}{\PYZsq{}}\PY{p}{]} \PY{o}{=} \PY{l+m+mi}{1}
         
         \PY{n}{df2}\PY{p}{[}\PY{p}{[}\PY{l+s+s1}{\PYZsq{}}\PY{l+s+s1}{control}\PY{l+s+s1}{\PYZsq{}}\PY{p}{,} \PY{l+s+s1}{\PYZsq{}}\PY{l+s+s1}{treatment}\PY{l+s+s1}{\PYZsq{}}\PY{p}{]}\PY{p}{]} \PY{o}{=} \PY{n}{pd}\PY{o}{.}\PY{n}{get\PYZus{}dummies}\PY{p}{(}\PY{n}{df2}\PY{p}{[}\PY{l+s+s1}{\PYZsq{}}\PY{l+s+s1}{group}\PY{l+s+s1}{\PYZsq{}}\PY{p}{]}\PY{p}{)}
\end{Verbatim}


    \begin{enumerate}
\def\labelenumi{\alph{enumi}.}
\setcounter{enumi}{2}
\tightlist
\item
  Use \textbf{statsmodels} to import your regression model. Instantiate
  the model, and fit the model using the two columns you created in part
  \textbf{b.} to predict whether or not an individual converts.
\end{enumerate}

    \begin{Verbatim}[commandchars=\\\{\}]
{\color{incolor}In [{\color{incolor}37}]:} \PY{k+kn}{import} \PY{n+nn}{statsmodels}\PY{n+nn}{.}\PY{n+nn}{api} \PY{k}{as} \PY{n+nn}{sm}
         
         \PY{n}{logit} \PY{o}{=} \PY{n}{sm}\PY{o}{.}\PY{n}{Logit}\PY{p}{(}\PY{n}{df2}\PY{p}{[}\PY{l+s+s1}{\PYZsq{}}\PY{l+s+s1}{converted}\PY{l+s+s1}{\PYZsq{}}\PY{p}{]}\PY{p}{,} \PY{n}{df2}\PY{p}{[}\PY{p}{[}\PY{l+s+s1}{\PYZsq{}}\PY{l+s+s1}{intercept}\PY{l+s+s1}{\PYZsq{}}\PY{p}{,} \PY{l+s+s1}{\PYZsq{}}\PY{l+s+s1}{treatment}\PY{l+s+s1}{\PYZsq{}}\PY{p}{]}\PY{p}{]}\PY{p}{)}
         \PY{n}{results} \PY{o}{=} \PY{n}{logit}\PY{o}{.}\PY{n}{fit}\PY{p}{(}\PY{p}{)}
\end{Verbatim}


    \begin{Verbatim}[commandchars=\\\{\}]
Optimization terminated successfully.
         Current function value: 0.366118
         Iterations 6

    \end{Verbatim}

    \begin{enumerate}
\def\labelenumi{\alph{enumi}.}
\setcounter{enumi}{3}
\tightlist
\item
  Provide the summary of your model below, and use it as necessary to
  answer the following questions.
\end{enumerate}

    \begin{Verbatim}[commandchars=\\\{\}]
{\color{incolor}In [{\color{incolor}38}]:} \PY{n}{results}\PY{o}{.}\PY{n}{summary}\PY{p}{(}\PY{p}{)}
\end{Verbatim}


\begin{Verbatim}[commandchars=\\\{\}]
{\color{outcolor}Out[{\color{outcolor}38}]:} <class 'statsmodels.iolib.summary.Summary'>
         """
                                    Logit Regression Results                           
         ==============================================================================
         Dep. Variable:              converted   No. Observations:               290585
         Model:                          Logit   Df Residuals:                   290583
         Method:                           MLE   Df Model:                            1
         Date:                Fri, 30 Mar 2018   Pseudo R-squ.:               8.085e-06
         Time:                        03:07:22   Log-Likelihood:            -1.0639e+05
         converged:                       True   LL-Null:                   -1.0639e+05
                                                 LLR p-value:                    0.1897
         ==============================================================================
                          coef    std err          z      P>|z|      [0.025      0.975]
         ------------------------------------------------------------------------------
         intercept     -1.9888      0.008   -246.669      0.000      -2.005      -1.973
         treatment     -0.0150      0.011     -1.312      0.190      -0.037       0.007
         ==============================================================================
         """
\end{Verbatim}
            
    \begin{enumerate}
\def\labelenumi{\alph{enumi}.}
\setcounter{enumi}{4}
\tightlist
\item
  What is the p-value associated with \textbf{ab\_page}? Why does it
  differ from the value you found in the \textbf{Part II}?
  \textbf{Hint}: What are the null and alternative hypotheses associated
  with your regression model, and how do they compare to the null and
  alternative hypotheses in the \textbf{Part II}?
\end{enumerate}

    The null hypothesis is H0: p\_new = p\_old

Alternative Hypothesis is H1: p\_new is not equal to p\_old

The p-value associated with ab\_page is 0.1897 because of the Hypothesis
we have chose. It considers the `two-sided' approach to calculate the
p-value. In the Part II we are using `smaller than' approach. Hence both
the p-values are different.

    \begin{enumerate}
\def\labelenumi{\alph{enumi}.}
\setcounter{enumi}{5}
\tightlist
\item
  Now, you are considering other things that might influence whether or
  not an individual converts. Discuss why it is a good idea to consider
  other factors to add into your regression model. Are there any
  disadvantages to adding additional terms into your regression model?
\end{enumerate}

    The other factors needs to be considered in the regression model becasue
we can't just look at the aggregates for individual converts as there
will be some dicrimnating factors here. If we can look into each smaller
part or group of individuals then it might not be the same case and we
could see different results.

There will be disadvantes if we add too many terms and we would not get
convergence.

    \begin{enumerate}
\def\labelenumi{\alph{enumi}.}
\setcounter{enumi}{6}
\tightlist
\item
  Now along with testing if the conversion rate changes for different
  pages, also add an effect based on which country a user lives. You
  will need to read in the \textbf{countries.csv} dataset and merge
  together your datasets on the approporiate rows.
  \href{https://pandas.pydata.org/pandas-docs/stable/generated/pandas.DataFrame.join.html}{Here}
  are the docs for joining tables.
\end{enumerate}

Does it appear that country had an impact on conversion? Don't forget to
create dummy variables for these country columns - \textbf{Hint: You
will need two columns for the three dummy varaibles.} Provide the
statistical output as well as a written response to answer this
question.

    \begin{Verbatim}[commandchars=\\\{\}]
{\color{incolor}In [{\color{incolor}39}]:} \PY{n}{countries\PYZus{}df} \PY{o}{=} \PY{n}{pd}\PY{o}{.}\PY{n}{read\PYZus{}csv}\PY{p}{(}\PY{l+s+s1}{\PYZsq{}}\PY{l+s+s1}{countries.csv}\PY{l+s+s1}{\PYZsq{}}\PY{p}{)}
         \PY{n}{countries\PYZus{}df}\PY{o}{.}\PY{n}{head}\PY{p}{(}\PY{p}{)}
         \PY{n}{df\PYZus{}new} \PY{o}{=} \PY{n}{countries\PYZus{}df}\PY{o}{.}\PY{n}{set\PYZus{}index}\PY{p}{(}\PY{l+s+s1}{\PYZsq{}}\PY{l+s+s1}{user\PYZus{}id}\PY{l+s+s1}{\PYZsq{}}\PY{p}{)}\PY{o}{.}\PY{n}{join}\PY{p}{(}\PY{n}{df2}\PY{o}{.}\PY{n}{set\PYZus{}index}\PY{p}{(}\PY{l+s+s1}{\PYZsq{}}\PY{l+s+s1}{user\PYZus{}id}\PY{l+s+s1}{\PYZsq{}}\PY{p}{)}\PY{p}{,} \PY{n}{how}\PY{o}{=}\PY{l+s+s1}{\PYZsq{}}\PY{l+s+s1}{inner}\PY{l+s+s1}{\PYZsq{}}\PY{p}{)}
         \PY{n}{df\PYZus{}new}\PY{o}{.}\PY{n}{head}\PY{p}{(}\PY{p}{)}
\end{Verbatim}


\begin{Verbatim}[commandchars=\\\{\}]
{\color{outcolor}Out[{\color{outcolor}39}]:}         country                   timestamp      group landing\_page  \textbackslash{}
         user\_id                                                               
         630000       US  2017-01-19 06:26:06.548941  treatment     new\_page   
         630001       US  2017-01-16 03:16:42.560309  treatment     new\_page   
         630002       US  2017-01-19 19:20:56.438330    control     old\_page   
         630003       US  2017-01-12 10:09:31.510471  treatment     new\_page   
         630004       US  2017-01-18 20:23:58.824994  treatment     new\_page   
         
                  converted  intercept  control  treatment  
         user\_id                                            
         630000           0          1        0          1  
         630001           1          1        0          1  
         630002           0          1        1          0  
         630003           0          1        0          1  
         630004           0          1        0          1  
\end{Verbatim}
            
    \begin{Verbatim}[commandchars=\\\{\}]
{\color{incolor}In [{\color{incolor}40}]:} \PY{n}{df\PYZus{}new}\PY{p}{[}\PY{l+s+s1}{\PYZsq{}}\PY{l+s+s1}{country}\PY{l+s+s1}{\PYZsq{}}\PY{p}{]}\PY{o}{.}\PY{n}{value\PYZus{}counts}\PY{p}{(}\PY{p}{)}
         \PY{n}{df\PYZus{}new}\PY{p}{[}\PY{p}{[}\PY{l+s+s1}{\PYZsq{}}\PY{l+s+s1}{CA}\PY{l+s+s1}{\PYZsq{}}\PY{p}{,} \PY{l+s+s1}{\PYZsq{}}\PY{l+s+s1}{US}\PY{l+s+s1}{\PYZsq{}}\PY{p}{]}\PY{p}{]} \PY{o}{=} \PY{n}{pd}\PY{o}{.}\PY{n}{get\PYZus{}dummies}\PY{p}{(}\PY{n}{df\PYZus{}new}\PY{p}{[}\PY{l+s+s1}{\PYZsq{}}\PY{l+s+s1}{country}\PY{l+s+s1}{\PYZsq{}}\PY{p}{]}\PY{p}{)}\PY{p}{[}\PY{p}{[}\PY{l+s+s1}{\PYZsq{}}\PY{l+s+s1}{CA}\PY{l+s+s1}{\PYZsq{}}\PY{p}{,}\PY{l+s+s1}{\PYZsq{}}\PY{l+s+s1}{US}\PY{l+s+s1}{\PYZsq{}}\PY{p}{]}\PY{p}{]}
\end{Verbatim}


    \begin{Verbatim}[commandchars=\\\{\}]
{\color{incolor}In [{\color{incolor}41}]:} \PY{n}{countries\PYZus{}logit} \PY{o}{=} \PY{n}{sm}\PY{o}{.}\PY{n}{Logit}\PY{p}{(}\PY{n}{df\PYZus{}new}\PY{p}{[}\PY{l+s+s1}{\PYZsq{}}\PY{l+s+s1}{converted}\PY{l+s+s1}{\PYZsq{}}\PY{p}{]}\PY{p}{,} \PY{n}{df\PYZus{}new}\PY{p}{[}\PY{p}{[}\PY{l+s+s1}{\PYZsq{}}\PY{l+s+s1}{intercept}\PY{l+s+s1}{\PYZsq{}}\PY{p}{,} \PY{l+s+s1}{\PYZsq{}}\PY{l+s+s1}{CA}\PY{l+s+s1}{\PYZsq{}}\PY{p}{,} \PY{l+s+s1}{\PYZsq{}}\PY{l+s+s1}{US}\PY{l+s+s1}{\PYZsq{}}\PY{p}{]}\PY{p}{]}\PY{p}{)}
         \PY{n}{countries\PYZus{}results} \PY{o}{=} \PY{n}{countries\PYZus{}logit}\PY{o}{.}\PY{n}{fit}\PY{p}{(}\PY{p}{)}
\end{Verbatim}


    \begin{Verbatim}[commandchars=\\\{\}]
Optimization terminated successfully.
         Current function value: 0.366115
         Iterations 6

    \end{Verbatim}

    \begin{enumerate}
\def\labelenumi{\alph{enumi}.}
\setcounter{enumi}{7}
\tightlist
\item
  Though you have now looked at the individual factors of country and
  page on conversion, we would now like to look at an interaction
  between page and country to see if there significant effects on
  conversion. Create the necessary additional columns, and fit the new
  model.
\end{enumerate}

Provide the summary results, and your conclusions based on the results.

    \begin{Verbatim}[commandchars=\\\{\}]
{\color{incolor}In [{\color{incolor}42}]:} \PY{n}{countries\PYZus{}results}\PY{o}{.}\PY{n}{summary}\PY{p}{(}\PY{p}{)}
\end{Verbatim}


\begin{Verbatim}[commandchars=\\\{\}]
{\color{outcolor}Out[{\color{outcolor}42}]:} <class 'statsmodels.iolib.summary.Summary'>
         """
                                    Logit Regression Results                           
         ==============================================================================
         Dep. Variable:              converted   No. Observations:               290585
         Model:                          Logit   Df Residuals:                   290582
         Method:                           MLE   Df Model:                            2
         Date:                Fri, 30 Mar 2018   Pseudo R-squ.:               1.521e-05
         Time:                        03:12:15   Log-Likelihood:            -1.0639e+05
         converged:                       True   LL-Null:                   -1.0639e+05
                                                 LLR p-value:                    0.1983
         ==============================================================================
                          coef    std err          z      P>|z|      [0.025      0.975]
         ------------------------------------------------------------------------------
         intercept     -1.9868      0.011   -174.174      0.000      -2.009      -1.964
         CA            -0.0507      0.028     -1.786      0.074      -0.106       0.005
         US            -0.0099      0.013     -0.746      0.456      -0.036       0.016
         ==============================================================================
         """
\end{Verbatim}
            
    Here we can conclude that from the Logistic regression model also we
don't find any substantial difference between the old pages and new
pages but old pages are slightly better. We can accept the Null
hypothesis and retain the old page.

    Resources used:
https://stackoverflow.com/questions/14657241/how-do-i-get-a-list-of-all-the-duplicate-items-using-pandas-in-python

Udacity lecture videos and quiz notebooks

https://www.youtube.com/watch?v=7FTp9JJ5DfE\&feature=youtu.be

     \#\# Conclusions

Congratulations on completing the project!

\hypertarget{gather-submission-materials}{%
\subsubsection{Gather Submission
Materials}\label{gather-submission-materials}}

Once you are satisfied with the status of your Notebook, you should save
it in a format that will make it easy for others to read. You can use
the \textbf{File -\textgreater{} Download as -\textgreater{} HTML
(.html)} menu to save your notebook as an .html file. If you are working
locally and get an error about ``No module name'', then open a terminal
and try installing the missing module using
\texttt{pip\ install\ \textless{}module\_name\textgreater{}} (don't
include the ``\textless{}'' or ``\textgreater{}'' or any words following
a period in the module name).

You will submit both your original Notebook and an HTML or PDF copy of
the Notebook for review. There is no need for you to include any data
files with your submission. If you made reference to other websites,
books, and other resources to help you in solving tasks in the project,
make sure that you document them. It is recommended that you either add
a ``Resources'' section in a Markdown cell at the end of the Notebook
report, or you can include a \texttt{readme.txt} file documenting your
sources.

\hypertarget{submit-the-project}{%
\subsubsection{Submit the Project}\label{submit-the-project}}

When you're ready, click on the ``Submit Project'' button to go to the
project submission page. You can submit your files as a .zip archive or
you can link to a GitHub repository containing your project files. If
you go with GitHub, note that your submission will be a snapshot of the
linked repository at time of submission. It is recommended that you keep
each project in a separate repository to avoid any potential confusion:
if a reviewer gets multiple folders representing multiple projects,
there might be confusion regarding what project is to be evaluated.

It can take us up to a week to grade the project, but in most cases it
is much faster. You will get an email once your submission has been
reviewed. If you are having any problems submitting your project or wish
to check on the status of your submission, please email us at
dataanalyst-project@udacity.com. In the meantime, you should feel free
to continue on with your learning journey by continuing on to the next
module in the program.


    % Add a bibliography block to the postdoc
    
    
    
    \end{document}
